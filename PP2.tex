%==================================================
%      PREAMBOLO e DICHIARAZIONI INIZIALI
%==================================================
\documentclass[10pt,oneside,a4paper]{article}

\usepackage[latin1]{inputenc} 
\usepackage[italian]{babel}
\usepackage[T1]{fontenc}
\usepackage{siunitx} %Inserisce automaticamente i dati con le unit�  di misura correttamente formattate del SI (utilizzo: \SI{0.82}{m^2}, in generale \SI{misura con il punto decimale}{unit�  di misura})
\sisetup{output-decimal-marker = {.}, separate-uncertainty = true, input-uncertainty-signs = \pm, detect-weight=true, detect-family=true} %per usare SI con il punto decimale
\usepackage{listings} %Per citare codice informatico formattandolo correttamente
\usepackage{amsmath,amsthm,verbatim,amssymb,amsfonts,amscd, graphicx,mathtools}
\usepackage[makeroom]{cancel}
\newcommand{\abs}[1]{\left\lvert\,#1\,\right\rvert}
\usepackage{geometry}
\usepackage{epigraph}
\usepackage{booktabs}	%tabelle migliorate
\usepackage{tablefootnote}	%note a pi� di pagina in tabella
\usepackage{threeparttable} %tabella con note a pi� di tabella
\usepackage{caption}	%descrizione per figure
\usepackage{dblfnote}
\captionsetup{tableposition=top,figureposition=bottom,font=small} %setup descrizione
\usepackage{float}
\usepackage{esvect} %vettori
\usepackage{longtable} %tabelle lunghe
\usepackage[dvipsnames]{xcolor}
\definecolor{sepia}{HTML}{80002A}
\usepackage[colorlinks=true, citecolor=black, linkcolor=sepia, urlcolor=black]{hyperref}
\usepackage{mathrsfs}
\usepackage{circuitikz}


\usepackage{multicol}
\newenvironment{Figure}
  {\par\medskip\noindent\minipage{\linewidth}}
  {\endminipage\par\medskip}

\newcommand{\var}{\operatorname{var}}
\newcommand{\cov}{\operatorname{cov}}


\usepackage{listings} %Per inserire codice
\lstnewenvironment{codice_c}[1][]
{\lstset{basicstyle=\small\ttfamily, columns=fullflexible,
keywordstyle=\color{red}\bfseries, commentstyle=\color{blue},
language=C, basicstyle=\small,
numbers=left, numberstyle=\tiny,
stepnumber=2, numbersep=5pt, frame=shadowbox,  showstringspaces=false, #1}}{}

%==================================================
%                  PRIMA PAGINA
%==================================================

\title{\textsc{\textbf{Esercitazione 2}: Amplificatore ad emettitore comune}}
\author{\small{G. Galbato Muscio} \and \small{L. Gravina} \and \small{L. Graziotto}}
\date{\today}

\begin{document}
	\begin{figure}
		\centering
		\includegraphics[scale=0.5, trim={2.8cm 8.9cm 0 9cm}, clip]{logo.png}
	\end{figure}
	\maketitle
	\begin{center} 
		\fbox{{\fontsize{12pt}{8mm}\textsc{Gruppo 11}}} \\
	\end{center}
\hrule
\vspace{0.5cm}
\renewcommand{\abstractname}{Abstract}
\begin{abstract}
Si utilizza un transistor 2N2222A di tipo \emph{npn} per realizzare un amplificatore ad emettitore comune, con amplificazione di tensione di circa $A_v = -50$. Se ne studia quindi la risposta in frequenza e le resistenze in uscita e in ingresso.
\end{abstract}
\vspace{4cm}
\tableofcontents %Indice
\newpage


\pagebreak
\begin{multicols}{2}
%==================================================
%      		PROGETTO RETE AUTOPOLARIZZANTE
%==================================================
\section{Progetto della rete autopolarizzante}
Si realizza il circuito seguente per l'amplificatore, utilizzando un transistor 2N2222A di tipo \emph{npn}.

\begin{flushleft}
\hspace*{-1cm}
\begin{circuitikz} [scale=.9]
	\draw (0,0) node[ground]{}
	(0,0) to[sV, l=$v_{\text{s}}$] (0,2) to[R=$R_s$] (1.5,2) to[C=$C_1$] (3.5,2)
	(3.5,-0.5) to[R=$R_2$] (3.5,2)
	(3.5,2) to[R=$R_1$] (3.5,4.5)
	(5,2) node[npn] (npn) {}
	(3.5,2) -- (npn.B);
	\draw (5,-0.5) to[R=$R_E$] (npn.E);
	\draw (npn.C) to[R=$R_C$] (5,4.5);
	\draw (npn.C) to[C=$C_2$] ++(1.5,0) to[short, -*] ++(0.25,0) node[right] {$v_o$};
	\draw (npn.E) -- ++(1,0) to[C=$C_E$] (6,-0.5)
	(6,-0.5) -- (5,-0.5)
	(3.5,4.5) -- (5,4.5)
	(3.5,-0.5) -- (5,-0.5)
	(4.25,4.5) to[short, *-] (4.25,4.75) node[above] {$V_\text{CC}$}
	(4.25,-0.5) node[ground]{}
	;\end{circuitikz}
\end{flushleft}

I valori degli elementi utilizzati sono, come da misura con il multimetro e con il ponte:
\begin{equation*}
\begin{aligned}
R_1 &= \SI{111 \pm 111}{\ohm} \\
R_2 &= \SI{111 \pm 111}{\ohm} \\
R_C &= \SI{111 \pm 111}{\ohm} \\
R_E &= \SI{111 \pm 111}{\ohm} \\
V_\text{CC} &= \SI{111 \pm 111}{V} \\
C_1 &= \SI{111 \pm 111}{F} \\
C_2 &= \SI{111 \pm 111}{F} \\
C_E &= \SI{111 \pm 111}{F}.
\end{aligned}
\end{equation*}
Si verifica con il multimetro che le tensioni tra i diversi nodi del circuito siano compatibili con quelle previste dalla teoria, al fine di verificare il corretto funzionamento del circuito stesso. Si riporta in tabella~\ref{tab:verifica} il confronto tra valore previsto e misurato.

\begin{center}
\captionof{table}{Valori previsti e misurati per il circuito}
\label{tab:verifica}
\begin{tabular}{l|c|c}
 & Valore previsto & Valore misurato \\
\hline
$V_C$ & \SI{111}{V} & \SI{111 \pm 111}{V}\\
$V_B$ & \SI{111}{V} & \SI{111 \pm 111}{V}\\
$V_E$ & \SI{111}{V} & \SI{111 \pm 111}{V}\\
$V_\text{CE}$ & \SI{111}{V} & \SI{111 \pm 111}{V}\\
$V_\text{BE}$ & \SI{111}{V} & \SI{111 \pm 111}{V}\\
$I_C$ & \SI{111}{mA} & \SI{111 \pm 111}{mA}\\
$I_E$ & \SI{111}{mA} & \SI{111 \pm 111}{mA}\\
\hline
\end{tabular}
\end{center}

Si ha 



\end{multicols}
\end{document}